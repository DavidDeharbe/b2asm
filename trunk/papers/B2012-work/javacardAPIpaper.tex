\documentclass{entcs} 
\usepackage{entcsmacro}
\usepackage{graphicx}
\sloppy
% The following is enclosed to allow easy detection of differences in
% ascii coding.
% Upper-case    A B C D E F G H I J K L M N O P Q R S T U V W X Y Z
% Lower-case    a b c d e f g h i j k l m n o p q r s t u v w x y z
% Digits        0 1 2 3 4 5 6 7 8 9
% Exclamation   !           Double quote "          Hash (number) #
% Dollar        $           Percent      %          Ampersand     &
% Acute accent  '           Left paren   (          Right paren   )
% Asterisk      *           Plus         +          Comma         ,
% Minus         -           Point        .          Solidus       /
% Colon         :           Semicolon    ;          Less than     <
% Equals        =3D           Greater than >          Question mark ?
% At            @           Left bracket [          Backslash     \
% Right bracket ]           Circumflex   ^          Underscore    _
% Grave accent  `           Left brace   {          Vertical bar  |
% Right brace   }           Tilde        ~

% A couple of exemplary definitions:

\newcommand{\Nat}{{\mathbb N}}
\newcommand{\Real}{{\mathbb R}}

\def\lastname{Martins Moreira}

\begin{document}
\begin{frontmatter}
%  \title{Modeling the Java Card API (or Interface?) in B} \author{A. Martins Moreira\thanksref{ALL}\thanksref{anamaria}}
%  \address{DIMAp-UFRN, Natal, RN, Brazil} \author{S. de Oliveira Santos \thanksref{simone}}
%  \address{DIMAp-UFRN, Natal, RN, Brazil} \author{B. Emerson Gurgel Gomes \thanksref{bruno}}
%    \address{IFRN, Currais Novos, RN, Brazil} \author{D. Boris Paul D\'{e}harbe \thanksref{david}}
%      \address{DIMAp-UFRN, Natal, RN, Brazil} \thanks[ALL]{INES-CAPES-projeto GD?... o que mais?} \thanks[anamaria]{Email:
%    \href{mailto:anamaria@dimap.ufrn.br} {\texttt{\normalshape
%        anamaria@dimap.ufrn.br}}} 
%      \thanks[simone]{Email:
%    \href{mailto:simone82@ppgsc.ufrn.br} {\texttt{\normalshape
%        simone82@ppgsc.ufrn.br}}}
%      \thanks[bruno]{Email:
%    \href{mailto:bruno.gurgel@ifrn.edu.br} {\texttt{\normalshape
%        bruno.gurgel@ifrn.edu.br}}}
%       \thanks[david]{Email:
%    \href{mailto:david@dimap.ufrn.br} {\texttt{\normalshape
%        david@dimap.ufrn.br}}}

  \title{Modeling the Java Card API (or Interface?) in B} \author[ufrn]{A. Martins Moreira\thanksref{ALL}\thanksref{anamaria}}
  \author[ufrn]{S. de Oliveira Santos \thanksref{simone}}
  \author[ifrn]{B. Emerson Gurgel Gomes \thanksref{bruno}}
  \author[ufrn]{D. Boris Paul D\'{e}harbe \thanksref{david}}
  \address[ufrn]{DIMAp-UFRN, Natal, RN, Brazil} \thanks[ALL]{INES-CAPES-projeto GD?... o que mais? REUNI?}  \address[ifrn]{IFRN, Currais Novos, RN, Brazil} 
 \thanks[anamaria]{Email:
    \href{mailto:anamaria@dimap.ufrn.br} {\texttt{\normalshape
        anamaria@dimap.ufrn.br}}} 
      \thanks[simone]{Email:
    \href{mailto:simone82@ppgsc.ufrn.br} {\texttt{\normalshape
        simone82@ppgsc.ufrn.br}}}
      \thanks[bruno]{Email:
    \href{mailto:bruno.gurgel@ifrn.edu.br} {\texttt{\normalshape
        bruno.gurgel@ifrn.edu.br}}}
       \thanks[david]{Email:
    \href{mailto:david@dimap.ufrn.br} {\texttt{\normalshape
        david@dimap.ufrn.br}}}


% AMM: Precisa ser arrumado: The ENTCS macro package provides two alternatives to listing authors
% names and addresses.  These are described in detail in the file
% \texttt{instraut.dvi}. Basically, listing each author and his or her
% address in turn, is the simplest method.  But, if there are several
% authors and two or more share the same address (but not all authors
% are at this address), then the method of listing authors first, and
% then the addresses, and of referencing addresses to authors should be
% used.


\begin{abstract} 
  The B method development process starts from an abstract model of a
  set of services that are to be developed, and includes formally
  verified refinements, which introduce implementation details. At a
  final refinement step these implementation details depend on the
  language for which code is to be generated. To rigorously develop
  Java Card Applications with the B method, then, it is necessary to
  model in B the constructs of Java Card, so that the gap between the
  B formally verified implementation model and the Java Card generated
  code is as small as it can be. In the context of the BSmart project,
  we have developed the specification of the Java Card Application
  Programming Interface (Java Card API), to be used in the development
  of Java Card applications with B.
\end{abstract}
\begin{keyword}
  Java Card, smart cards, B method, formal methods
\end{keyword}
\end{frontmatter}
\section{Introduction}\label{intro}

In the context of the BSmart project, we are interested in the use of
B to develop Java Card~\cite{chen:2000} applications.  Java Card is a
restricted and optimized version of Java which allows memory and
processor constrained devices, such as smart cards, to store and run
small applications.  Java Card applications have characteristics
(small size and need for security) which make then good candidates for
formal development with the B method. They also present some
regularity of structure that induced our group into the definition of
a method and a tool which specialize B for development of such
applications, providing greater support for the developer than pure B
tools \cite{atelierB,proB}. They are the BSmart method and tool
presented in \cite{Gomes10}.
%colocar o de 2008 tamb�m

Previous works have also dealt with B for Java Card, 
\cite{BOM,JBtools,e cia}, but none has, at our knowledge, specified Java Card
in B so that the formal development includes Java Card aspects. In the
first versions of BSmart \cite{Gomes10}, these aspects were included by the code
ganerator, and in JBTools \cite{JBtools}, the generated code needed to be edited by
the developer to include these aspects.

These works lacked then the ability to formally verify the Java Card
dependent aspects of the application. To be able to change this, a B
specification of these aspects is needed, and this contribution is
what we present in this paper: the B specification of the Java Card
API, with...  This specification can be used with the BSmart tool,
independently, in any B tool, or even as a complement to the official
Java Card documentation \cite{oracle}.

In the following...

\section{Smart Cards and Java Card}\label{sec:javacard}


AMM: TRECHOS EXTRAIDOS E JA UM POUCO ALTERADOS DO ABZ 2010. DEPENDENDO
DO QUE VAMOS MOSTRAR NA ESPECIFICA�AO PODEMOS DEIXAR MAIS OU MENOS
DETALHES AQUI. NO FINAL, PRECISAMOS VERIFICAR QUE NAO COPIAMOS DEMAIS
O QUE ESTAVA LA.

A smart card application is distributed between on-card and off-card
components.  The server application on the card side (called applet)
provides the application services. The off-card client (called host
application) resides in a computer or electronic terminal.

The information exchange between the host and card applications is
made through a a half-duplexed low level protocol, named
\emph{Application Protocol Data Unit} (APDU).  The ISO 7816-4 standard
specifies two kinds of APDU's, which are the command and the response
APDU (Figure~\ref{fig:APDUs}). Both specify data packets. A command APDU
is sent by the host, requiring some applet service and a response APDU is
sent by the applet, responding to the host request with the result of the
service processing.

\begin{figure}[!ht]
\centering
%\includegraphics[scale=0.3]{figs/comandoAPDU.jpg}
%\includegraphics[scale=0.3]{figs/respostaAPDU.jpg}
\caption{Command and response APDUs. Source:~\cite{ortiz1intro:2003}}
\label{fig:APDUs}
\end{figure}

The main component of the Java Card platform (Figure~\ref{fig:jcre})
is its runtime environment (JCRE), composed of a Java Card Virtual
Machine (JCVM), an API, and, usually, system and
industry-specific classes~\cite{chen:2000}. The JCRE acts as a small
operating system, being responsible for the control of the application
lifetime, security and resource management.

\begin{figure}[h]
\centering
%\includegraphics[width=0.35\textwidth]{figs/jcplatform.png}
\caption{Java Card platform components. Adapted from:~\cite{ortiz1intro:2003}}
\label{fig:jcre}
\end{figure}


\paragraph{Java Card applets}\label{sec:applets}

A Java Card applet is a class that inherits the
\emph{javacard.framewok.Applet} class of the Java Card API and is
implemented upon the Java Card subset of Java.  During the applet
conversion for card installation, a verification phase is performed to
check conformance of the classes with Java Card restrictions.

The current usual Java Card specification (2.2.x) allows two kinds of
applets.  The older, and most commonly used, kind of applet manipulates
directly the APDU protocol packages while the newer one abstracts from
the lower level protocol using Remote Method Invocation (RMI). In this
paper we will call the lower level applet \emph{APDU applet} and the
higher level one, \emph{RMI applet}. RMI introduces a layer of
abstraction above the APDU protocol, and, due to this fact, it is usually
less efficient than APDU applets.


\section{B specification of the Java Card API}




\section{Using the model in BSmart}


\section{Conclusions}



\bibliographystyle{plain}
\bibliography{bib}

\end{document}
