\section{B Method}

\begin{frame}
\hypertarget{metodoB}{}
  \frametitle{B Method} 
  \begin{itemize}
    \item B method for software development is a formal method based on
	    \begin{itemize}
	    \item  B Abstract Machine Notation (AMN)
	    \item  First order logic, integer arithmetic and set theory
	    \end{itemize}
    \item Its constructions are very similar to those of the Z notation    
    \item Supports:
    \begin{itemize}
	    \item Modularization
	    \item Refinements proved until basic constructs of programming language 
	    \item Advanced technicals to process prove 
	    \item Code generator to programming language
	    \item \ldots
    \end{itemize}   
  \end{itemize}
\end{frame}
    
    
    
\begin{frame}
\frametitle{B Method- Simple example}
\begin{figure}
  % \begin{small}
  $$
  \begin{array}{lcl}
    \begin{array}[t]{l}
      \MACHINE \\ \quad  \mathit{micro} \\
      \SEES \\ \quad \mathit{TYPES}, \mathit{ALU} \\
      \INCLUDES \\  \quad \mathit{MEMORY} \\
      \VARIABLES \\ \quad    \mathit{pc}  \\
      \INVARIANT \\ \quad \mathit{pc} \in \mathit{INSTRUCTION} 
    \end{array}
    & \hspace*{0cm} &
    \begin{array}[t]{l}
      \INITIALISATION  \mathit{pc} := 0 \\
      \OPERATIONS\\
      \mathit{JMP} \mathit{( jump )} = \\
      \quad \PRE \mathit{jump} \in \mathit{INSTRUCTION}\\ 
      \quad \THEN \mathit{pc} := \mathit{jump}\\  
      \quad \END \\
      \END
    \end{array}
  \end{array}
  $$
\caption{A very basic B machine.}
\label{fig:maqB}
\end{figure}
\end{frame}