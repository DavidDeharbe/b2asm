---------------------------------------------

Paper: 25 Title: A Formal Model of Microcontroller Instruction Set in B


-------------------- review 1 --------------------

PAPER: 25 TITLE: A Formal Model of Microcontroller Instruction Set in B

OVERALL RATING: 1 (weak accept) ----------------------- REVIEW
--------------------

The paper describes a B model of the Z80 microcontroller. The major goal of the
project is to be able to verify software at the assembly level.

The project, its current achievements and its scope seem to be adequate to the
thesis & dissertation workshop.

Suggestions:
 - The paper needs an extensive and careful review of the English
language. 
[Res: Vou levar p/ um outro revisor]
 - The authors must look at other previous work on formal verification
of microcontrollers, processors, architectures and assembly code (regardless the
formal language used and the processor verified). For instance, the work by
Anthony Fox in the verification of ARM can be a good starting point. His
technical reports cite many other related work (Sawada and Hunt, Tahar and Kumar,
Windley and Coe, etc.)
[Res: Vou levar p/ um outro revisor]
 - Sections 6 and 7 contain several claims that must be better explained. "this paper quoted some
 techniques to lower the cost of modelling" (which ones?) "This work specify a model more simple to get better
results, at the same time, the model stay near to model of actual
microcontrollers." (why is your model simpler? what makes it simpler and still
closer to the microcontrollers in comparison to previous work?) "B method has an
easy notation, similar to the Pascal language" (I don't think B and Pascal are
similar) 
[Res: V]
- "we found some errors and ambiguities in the official manual" This is
an excellent contribution that the paper overlooks. - In reference 4, do not use
"Et al."


-------------------- review 2 --------------------

PAPER: 25 TITLE: A Formal Model of Microcontroller Instruction Set in B

OVERALL RATING: 2 (accept) ----------------------- REVIEW --------------------

The paper describes an interesting encoding of the Z80 microcontroller model
using the B method. This encoding is useful for proving a number of properties
about the microcontroller. They found some errors and ambiguities in the
microcontroller official manual. This is a very nice result and shows the
importance of formally specifying and proving things using a theorem prover.



However, the paper is not well-written. Run the spell checker and rewrite some
sentences. In the abstract and introduction, you should describe more about the
context and show the importance of formalizing the Z80 model.
[Res: Não encontrei erros reais no spell checker, mas vou levar o trabalho para um revisor!]
[O contexto é a documentação das instruções/especificação e a principal importancia é a verificação em]
[nível de assembly. Tudo isso é esclarecido de forma suscinta devido ao espaço. ]
Additionally, in the conclusions, discuss more and describe the experience of formalizing the
model and proving properties of it. It would be nice if you give more details
about your dissertation (schedule, the main goal,…).
[Res: A experiencia sobre o processo de prova é descrito suscintamente na seção Proofs]
[Vou enfatizar que os trabalhos futuros será elaborado na dissertação]



(page 1) “… two model microcontrollers that are under construction …” Do you
reuse some parts of the model? And proofs? Evolving specifications is not an easy
task since some previous proofs may be proved again, as you mention in Section 5
(you need to reprove some lemmas). A simple renaming may have an impact on
several proofs. Does the B tool have some support in this scenario? Do you have
some guidelines in order to minimize the impact during evolution?
[Res - Vou adicionar: Esses projetos estão estruturado de um modo que pode facilmente]
[ ser reusado por um plataforma diferente ]



(page 3) “… we have been developed a reusable set of basic definitions …” A
well-structured specification will allow you to improve reusability (and also
reuse some proofs). Do you structure your specification in a way that can
minimize the impact?
[Res: Praticamente não deve existir impacto nas bibliotecas, quando uma nova plataforma é reusa isso. ]
[ A resposta anterior já trata isso!]
Does the B prover allow the user to define proof strategies (a user defined proof
command that is created based on other ones)?
[Res: O provador tem estratégia de descrever comandos que podem ser aplicados a várias oP diferente]
[,desde que tenha forma semelhantes. Inf] Proving theorems in a theorem prover is not an easy task, 
and your experience can be very useful. This is a good topic to explore more and discuss your experience.
[Res- Vou adicionar: Sim! Foi utilizado quatro computadores dual-core para realizar as provas  ]



Discuss whether the lemmas you proved were difficult. Does the B theorem prover
help you to prove most of them automatically? Show more details about the
automation level of the prover. Do you have some experience in other theorem
provers (Z-Eves, PVS, Isabelle,…)? In case you have, can you compare them to the
B prover?
[Res - Vou add: Sem usar comandos de prova, em geral 70 foi automatica  ,]
[Não tenho experiência em outros provadores]




(page 6) “… It supports 158 instructions …” Do you specify all of them?
[Res - Adicionar: YES! :D]

(page 8) “… Thus, some aspects are not modeled …” What is the impact of not
considering some aspects in the results of your proofs? You must give more
details about it. Moreover, you should precisely state all aspects that you do
not consider.
[Res - Adicionar: Por exemplo, tempo e detales architecture, porque eles não são relevantes para
verificação assembly]
What are the benefits of your approach for a system using this microcontroller?
[Res: O texto esclarece que é mais fácil que tentar inicialmente em um processo real]



(page 8) “… we found some errors and ambiguities in the official manual …”
Describe some kinds of errors. This is a very nice result and you must give much
more details.
[Res: O paper só tem oito páginas :( ]

It is important to mention some details about what are you going to do in your
dissertation, mention the main goal of it, and the results that you are aiming
at. It would be nice to precisely state the problem you are tackling.
[Res Adicionar: Vou retificar que os trabalhos futuros serão feitos na minha dissertação]



Minor comments (page 2) 
“… the Z notation …” -> missing (.) 
(page 2) 
“… Section 2 explain …” 
(page 2) “… the figure 1 …”  [NÃO ENTENDI ESSE COMENTÁRIO] 
(page 2) “… the its initial …” 
(page 6) “…from operations are or predefined…”

In Section 3, it is better to first explain the microcontroller and then what you
have done. In the beginning of each section, it is important to relate it with
the previous and following sections.
[Res: Como o espaço é pequeno e o modelo do microcontroller usa os conceitos das biblioteca,]
[Então, eu explico primeiro as blibliotecas que fiz, depois o modelo do microcontroller.]

(page 5) “… The type BV16 is created for bit vector of length 16 …” Where is this
type used? For each type declaration, give an intuition specifying which part of
the microcontroller it is representing.
[Res - Adicionar: Vou adicionar o módulo BV16 e BYTES na figura 1. Não entendi bem ]
[Mas todos os tipos citado são utilizados pelo microcontrolador.]